% Options for packages loaded elsewhere
\PassOptionsToPackage{unicode}{hyperref}
\PassOptionsToPackage{hyphens}{url}
%
\documentclass[
  man]{apa6}
\usepackage{amsmath,amssymb}
\usepackage{lmodern}
\usepackage{iftex}
\ifPDFTeX
  \usepackage[T1]{fontenc}
  \usepackage[utf8]{inputenc}
  \usepackage{textcomp} % provide euro and other symbols
\else % if luatex or xetex
  \usepackage{unicode-math}
  \defaultfontfeatures{Scale=MatchLowercase}
  \defaultfontfeatures[\rmfamily]{Ligatures=TeX,Scale=1}
\fi
% Use upquote if available, for straight quotes in verbatim environments
\IfFileExists{upquote.sty}{\usepackage{upquote}}{}
\IfFileExists{microtype.sty}{% use microtype if available
  \usepackage[]{microtype}
  \UseMicrotypeSet[protrusion]{basicmath} % disable protrusion for tt fonts
}{}
\makeatletter
\@ifundefined{KOMAClassName}{% if non-KOMA class
  \IfFileExists{parskip.sty}{%
    \usepackage{parskip}
  }{% else
    \setlength{\parindent}{0pt}
    \setlength{\parskip}{6pt plus 2pt minus 1pt}}
}{% if KOMA class
  \KOMAoptions{parskip=half}}
\makeatother
\usepackage{xcolor}
\IfFileExists{xurl.sty}{\usepackage{xurl}}{} % add URL line breaks if available
\IfFileExists{bookmark.sty}{\usepackage{bookmark}}{\usepackage{hyperref}}
\hypersetup{
  pdftitle={Sexism and Division of Labor on Feelings of Teamliness Within Household Dyads},
  pdfauthor={Naomi Liftman1 \& Cara Krupnikoff-Salkin1},
  pdflang={en-EN},
  pdfkeywords={keywords},
  hidelinks,
  pdfcreator={LaTeX via pandoc}}
\urlstyle{same} % disable monospaced font for URLs
\usepackage{graphicx}
\makeatletter
\def\maxwidth{\ifdim\Gin@nat@width>\linewidth\linewidth\else\Gin@nat@width\fi}
\def\maxheight{\ifdim\Gin@nat@height>\textheight\textheight\else\Gin@nat@height\fi}
\makeatother
% Scale images if necessary, so that they will not overflow the page
% margins by default, and it is still possible to overwrite the defaults
% using explicit options in \includegraphics[width, height, ...]{}
\setkeys{Gin}{width=\maxwidth,height=\maxheight,keepaspectratio}
% Set default figure placement to htbp
\makeatletter
\def\fps@figure{htbp}
\makeatother
\setlength{\emergencystretch}{3em} % prevent overfull lines
\providecommand{\tightlist}{%
  \setlength{\itemsep}{0pt}\setlength{\parskip}{0pt}}
\setcounter{secnumdepth}{-\maxdimen} % remove section numbering
% Make \paragraph and \subparagraph free-standing
\ifx\paragraph\undefined\else
  \let\oldparagraph\paragraph
  \renewcommand{\paragraph}[1]{\oldparagraph{#1}\mbox{}}
\fi
\ifx\subparagraph\undefined\else
  \let\oldsubparagraph\subparagraph
  \renewcommand{\subparagraph}[1]{\oldsubparagraph{#1}\mbox{}}
\fi
\newlength{\cslhangindent}
\setlength{\cslhangindent}{1.5em}
\newlength{\csllabelwidth}
\setlength{\csllabelwidth}{3em}
\newlength{\cslentryspacingunit} % times entry-spacing
\setlength{\cslentryspacingunit}{\parskip}
\newenvironment{CSLReferences}[2] % #1 hanging-ident, #2 entry spacing
 {% don't indent paragraphs
  \setlength{\parindent}{0pt}
  % turn on hanging indent if param 1 is 1
  \ifodd #1
  \let\oldpar\par
  \def\par{\hangindent=\cslhangindent\oldpar}
  \fi
  % set entry spacing
  \setlength{\parskip}{#2\cslentryspacingunit}
 }%
 {}
\usepackage{calc}
\newcommand{\CSLBlock}[1]{#1\hfill\break}
\newcommand{\CSLLeftMargin}[1]{\parbox[t]{\csllabelwidth}{#1}}
\newcommand{\CSLRightInline}[1]{\parbox[t]{\linewidth - \csllabelwidth}{#1}\break}
\newcommand{\CSLIndent}[1]{\hspace{\cslhangindent}#1}
\ifLuaTeX
\usepackage[bidi=basic]{babel}
\else
\usepackage[bidi=default]{babel}
\fi
\babelprovide[main,import]{english}
% get rid of language-specific shorthands (see #6817):
\let\LanguageShortHands\languageshorthands
\def\languageshorthands#1{}
% Manuscript styling
\usepackage{upgreek}
\captionsetup{font=singlespacing,justification=justified}

% Table formatting
\usepackage{longtable}
\usepackage{lscape}
% \usepackage[counterclockwise]{rotating}   % Landscape page setup for large tables
\usepackage{multirow}		% Table styling
\usepackage{tabularx}		% Control Column width
\usepackage[flushleft]{threeparttable}	% Allows for three part tables with a specified notes section
\usepackage{threeparttablex}            % Lets threeparttable work with longtable

% Create new environments so endfloat can handle them
% \newenvironment{ltable}
%   {\begin{landscape}\centering\begin{threeparttable}}
%   {\end{threeparttable}\end{landscape}}
\newenvironment{lltable}{\begin{landscape}\centering\begin{ThreePartTable}}{\end{ThreePartTable}\end{landscape}}

% Enables adjusting longtable caption width to table width
% Solution found at http://golatex.de/longtable-mit-caption-so-breit-wie-die-tabelle-t15767.html
\makeatletter
\newcommand\LastLTentrywidth{1em}
\newlength\longtablewidth
\setlength{\longtablewidth}{1in}
\newcommand{\getlongtablewidth}{\begingroup \ifcsname LT@\roman{LT@tables}\endcsname \global\longtablewidth=0pt \renewcommand{\LT@entry}[2]{\global\advance\longtablewidth by ##2\relax\gdef\LastLTentrywidth{##2}}\@nameuse{LT@\roman{LT@tables}} \fi \endgroup}

% \setlength{\parindent}{0.5in}
% \setlength{\parskip}{0pt plus 0pt minus 0pt}

% \usepackage{etoolbox}
\makeatletter
\patchcmd{\HyOrg@maketitle}
  {\section{\normalfont\normalsize\abstractname}}
  {\section*{\normalfont\normalsize\abstractname}}
  {}{\typeout{Failed to patch abstract.}}
\patchcmd{\HyOrg@maketitle}
  {\section{\protect\normalfont{\@title}}}
  {\section*{\protect\normalfont{\@title}}}
  {}{\typeout{Failed to patch title.}}
\makeatother
\shorttitle{Sexism and Feelings of Teamliness}
\keywords{keywords\newline\indent Word count: X}
\DeclareDelayedFloatFlavor{ThreePartTable}{table}
\DeclareDelayedFloatFlavor{lltable}{table}
\DeclareDelayedFloatFlavor*{longtable}{table}
\makeatletter
\renewcommand{\efloat@iwrite}[1]{\immediate\expandafter\protected@write\csname efloat@post#1\endcsname{}}
\makeatother
\usepackage{csquotes}
\ifLuaTeX
  \usepackage{selnolig}  % disable illegal ligatures
\fi

\title{Sexism and Division of Labor on Feelings of Teamliness Within Household Dyads}
\author{Naomi Liftman\textsuperscript{1} \& Cara Krupnikoff-Salkin\textsuperscript{1}}
\date{}


\affiliation{\vspace{0.5cm}\textsuperscript{1} Smith College}

\begin{document}
\maketitle

\hypertarget{introduction}{%
\section{Introduction}\label{introduction}}

For many decades, women in dual-earner households were found to do a significant portion of the housework (Bareket, Shnabel, Kende, Knab, \& Bar-Anan, 2020), thus creating a so-called ``second shift'' of work for women when they returned home. In May of 2020, as the devastation of the COVID-19 pandemic began to set in, nearly 48\% of the Americans that had previously commuted to work in February were either working from home or unemployed (Bick \& Mertens, 2020). As a result of the pandemic, the entire American workforce was rearranged, leading to a dual purpose of the current study. The first is to synthesize previous research findings on the relationship between sexism, marital satisfaction, and the division of housework in the context of romantic relationships, while determining whether these findings hold true during such unprecedented times. The second is to explore these concepts through a dyadic lens, by looking at the experiences of both partners within a heterosexual relationship.

\hypertarget{sexism-in-romantic-couples}{%
\subsection{Sexism in Romantic Couples}\label{sexism-in-romantic-couples}}

When looking at the mechanisms that play into the differences within and between heterosexual couples, it is important to acknowledge the different forms of sexism that occur. Under the theory of ambivalent sexism, there are two forms of sexism. The first is hostile sexism (HS): where women are treated with resent, and often aggression. The second is benevolent sexism (BS): where women are cared for, but seen as helpless and incompetent (Glick \& Fiske, 1996). Furthermore, BS and HS interact to create the structure of gender inequality that is present within our modern-day society. While HS punishes women for breaking traditional gender roles, BS rewards them for taking on traditionally feminine roles (Glick \& Fiske, 2001). As a result, ambivalent sexism has a strong influence on interpersonal relationships. Within a romantic couple, men who are higher in BS tend to seek women who fulfill more traditional gender roles (Thomae \& Houston, 2016). Moreover, men who are high in HS seek women who are high in BS, and vice versa (Lee, Fiske, Glick, \& Chen, 2010). Individuals who are higher in either form of sexism are more likely to have partnerships that fit more traditional gender prescriptions.

Additionally, Hammond and Overall (2013) found that women who were high in BS experienced very high relationship satisfaction if their benevolent ideals were met, but very low satisfaction if they weren't. Conversely, they found that a man's relationship satisfaction had more to do with his overall status (inside and outside the home) than on his partner's fulfillment of her gender role. Minnotte, Minnotte, and Pedersen (2013) show that the relationship between gender ideologies and marital satisfaction is complex. In their study, when husbands were lower in sexism and the household was more egalitarian, husbands reported less relationship satisfaction. This indicates that in addition to behaviors and reported beliefs, there are unseen factors that influence gender bias. Lee et al. (2010) also found that even when Americans don't directly endorse BS attitudes, they are still likely to be guided by BS when choosing their partners, highlighting how even in instances where people don't present bias outwardly, they are often still influenced by gender ideologies.

\hypertarget{dividing-housework}{%
\subsection{Dividing Housework}\label{dividing-housework}}

A prominent manner of observing bias from gender ideologies is through the division of household labor. Typically, women do more frequent, routine chores, such as cleaning, and men do more occasional, intermittent chores, such as car maintenance (Barstad, 2014). This leads to a misalignment, where even with increasing gender equality within the workforce, women are doing disproportionately more work (Helms, Walls, Crouter, \& McHale, 2010). It is important to note that calculations of housework are often biased. Stereotypical men and women engage in different types of chores, so any calculation based on a set list of chores may miss the larger picture of how the couple chooses to divide their work on any given day (Nordenmark \& Nyman, 2003). Regardless, focusing specifically on only traditionally feminine tasks allows us to highlight the gender mismatch in the proportion of everyday tasks taken on by different members of a household.

While dual-earning heterosexual households may have more egalitarian divisions of housework (Chesters, 2013), the divisions are still uneven. In fact, Barstad (2014) found that in the cases in which there is greater equality in the division of household labor, it is because the women have decreased their housework, as opposed to the men taking on more. This leads to an inequity where women and men have different concepts of what equal division of housework may mean. Additionally, division of housework is influenced heavily by the individual partners' work environments, especially when the partners are not dual earners. When husbands work or earn more than their wives, wives end up taking on a larger proportion of the housework (Lam, McHale, \& Crouter, 2012). Similarly, when husbands are dealing with stress from work that extends back to their home, the wife is more likely to take on more work (Huffman, Matthews, \& Irving, 2016; Lam et al., 2012). This last finding is particularly relevant to the context of the COVID-19 pandemic, in which many couples have been forced to work from home. In order to gain insight into the process through which chores are divided, it is important to determine how couples are responding to changes in status and the spillover of work-family conflict during these unprecedented times.

\hypertarget{sexism-and-marital-satisfaction-linked-with-division-of-housework}{%
\subsection{Sexism and Marital Satisfaction Linked with Division of Housework}\label{sexism-and-marital-satisfaction-linked-with-division-of-housework}}

Many of the factors that influence the ways in which housework is divided are linked closely with sexism. In general, when couples exhibit more traditional gender ideologies, the woman is likely to do more housework (Erickson, 2005). When one partner exhibits different sexist beliefs than the other, the division of household labor continues to be uneven (Bareket et al., 2020). In studying attitudes toward provider roles, Helms et al. (2010) found that those who endorse the main-secondary provider and ambivalent coprovider attitudes --- those that relate most closely to BS beliefs (Carlson \& Hans, 2020) --- the wife is more likely to do the majority of the housework. Doan and Quadlin (2018) found that it is much more common and acceptable for the wife to take on masculine tasks in addition to feminine (routine) ones than it is for the husband to take on any of the feminine tasks. This leads to an imbalance in which women are often left doing more work than men. At the same time, according to Poortman and Lippe (2009), one of the main underlying causes of unequal division of household labor is because women prefer to do the routine house chores. In many cases, the wife's prescriptions guide her to take on a traditional role within the household. Additionally, although the husband has traditionally had a more dominant role in determining the gender roles within the relationship (Huffman et al., 2016; Lam et al., 2012), the wife also has a strong influence on whether the division of housework will be more traditional or not. When the wife endorses more traditional gender ideologies, even if the husband doesn't, the wife is likely to do more housework (T. N. Greenstein, 1996). When the wife does not endorse BS, it is more likely for there to be an equal division of housework (Helms et al., 2010). While many studies have looked individually at the effect that the husband's and the wife's sexist attitudes have on their division of household labor, less have looked at \emph{how} the relationship between their gender ideologies may have an impact.

There is also a strong link between division of household labor and feelings of marital satisfaction. Feelings of fairness and equality within the partnership are more strongly linked to the distribution of housework than to the division of paid work (Nordenmark \& Nyman, 2003). In countries with more gender-equity, such as America, the distribution of household chores has a considerable effect on feelings of fairness (T. Greenstein, 2009). Interestingly, there are also strong relationships between the division of household labor and relationship quality in women. Women feel low levels of satisfaction when men do little to no routine housework, but also don't enjoy doing intermittent (masculine) tasks (Barstad, 2014). Coupled with gender ideologies, division of household labor has a strong impact on feelings of marital satisfaction. When the wife takes on less of a traditional role and participates more in the workforce, she is less likely to be satisfied with doing more housework (Braun, Lewin-Epstein, Stier, \& Baumgärtner, 2008). Similarly, people who endorse different sexist beliefs are likely to have different perceptions about what divisions of housework are equitable, influencing feelings about fairness, teamwork, and marital quality (Bareket et al., 2020; Nordenmark \& Nyman, 2003). It is rare for two partners to share the exact same gender ideologies, and gender ideologies have an impact on the division of housework and on overall marital satisfaction. Each individual person's beliefs have an impact on their own feelings, but also on their partner's, making it important to investigate these variables from a dyadic perspective.

\hypertarget{current-study}{%
\subsection{Current Study}\label{current-study}}

Previous research shows that there are links between sexism and division of household labor, sexism and marital satisfaction, and division of household labor and marital satisfaction. While some research looks at the relationship between all three of these variables, it fails to address the relationship that the two partners have with each other. In this study, we attempted to gain a more holistic understanding of how the interaction between partners influences their individual attitudes towards sexism, feelings about the quality of the relationship, and the percentage of housework that they are a part of. Couples were asked about their feelings about their relationship and the household chores that they did for two weeks, with the goals of (1) determining the extent to which the husbands' and the wives' attitudes of sexism impacted their own and each other's feelings of marital satisfaction, and (2) the extent to which division of labor may have acted as a moderator between the two variables. We approached the data dyadically, using the Actor-Partner Interdependence Model {[}APIM; Kenny, Kashy, Cook, and Simpson (2006){]}. This allowed us to operate under the assumption that the romantic partners are not independent from each other, and to investigate the effects both the respondents (actors) and their partners (partners) had on each other.

Our first hypothesis was that individuals who were higher in sexism would have higher feelings of marital satisfaction if their partners were also higher in sexism. Given the findings by Thomae and Houston (2016) and Lee et al. (2010), we hypothesized that this would be especially true for scenarios in which women were high in BS and men were high in either HS or BS (Hypothesis 1a). Our second hypothesis was that the relationship between sexism and marital quality would be moderated by the division of household labor. More specifically, we predicted that when both partners were higher in sexism, and the division of housework would be more traditional, with the woman doing more routine housework, causing both partners to have greater feelings of satisfaction (Hypothesis 2a). In instances where one partner was higher in sexism than the other, we predicted that the division of housework would be traditional, and that the man would be more satisfied (Hypothesis 2b); if the woman was high in sexism, she would be as satisfied or more satisfied than the man (Hypothesis 2c); if she was low in sexism, she would be less satisfied than the man (Hypothesis 2d). Lastly, if both partners were low in sexism and the division of housework was more egalitarian, both partners would be satisfied (Hypothesis 2e). While each of these hypotheses was drawn from empirical literature that described the relationship of these components on an individual level, we anticipated that there may be contradictory findings once we take into account the potential dyadic interaction within each couple for these variables.

\hypertarget{methods}{%
\section{Methods}\label{methods}}

\hypertarget{measures}{%
\subsection{Measures}\label{measures}}

\hypertarget{hostile-and-benevolent-sexism}{%
\subsubsection{Hostile and Benevolent Sexism}\label{hostile-and-benevolent-sexism}}

See table 1 for correlation matrix between sexism scores. Hostile sexism (HS) and benevolent sexism (BS) were measured using a modified version of the Ambivalent Sexism Inventory (ASI) created by Glick and Fiske (1997). The original scale includes 22 questions with 11 for both subcategories of sexism; however, we removed question 18, ``There are actually very few women who get a kick out of teasing men by seeming sexually available and then refusing male advances''. Participants did not appear to understand the wording of the question, and responses were inconsistent with their other answers. Therefore, our final measure includes 10 questions assessing HS and 11 assessing BS. For each measure we calculated an average, with a higher score for either scale indicating higher sexist beliefs. Alphas for HS and BS were .87 and .80, respectively and the intraclass correlations were .59 and .32.

\hypertarget{division-of-labor}{%
\subsubsection{Division of Labor}\label{division-of-labor}}

To calculate the division of household labor we asked each participant to complete 14 daily surveys, in which they were given a list of routine chores and were asked to indicate, ``Today, did you spend any time on the following household chores? (Yes or No)'' Since we are looking to isolate routine chores, we chose to omit intermittent chores from our calculation of housework. We removed the responses to the categories ``Took care of our cars today (sent to repairs, washing, vacuuming)'' and ``Prepared for events and activities today (for example, birthdays or anniversaries)''. We also chose not to include text responses of additional chores because most of them fell into other chore categories or were better described as intermittent. There were 11 remaining chores (e.g., laundry, cooking, yardwork, etc.), with which we calculated the proportion of chores performed by each partner. The proportion was calculated as the sum of chores one partner did on a given day divided by the sum of chores performed by both partners on a given day. We left this amount as a decimal, so if a partner performed all the chores for that day they had a score of one, and if they performed no chores they had a score of zero. So higher scores indicate a larger proportion of chores done and lower scores indicate a smaller proportion of chores performed on a given day. We also coded it so if no chores were performed on a certain day by either partner, both partners received a zero. The intraclass correlation was .38, which means that there is a relatively low correlation between partners chores performed per day. This sparked our interest, because it seems unlikely that the amount of chores one partner performs would not effect the amount of chores another partner performs. We believe this is likely due to the way we coded if no chores were performed, then both partners received a zero.

\hypertarget{teamliness}{%
\subsubsection{Teamliness}\label{teamliness}}

In order to assess feelings about working together as partners, participants were asked about the degree to which they identified with the statement: ``Today, my partner and I are really a team'' in their daily diaries. Their answers were scored on a four point scale, with possible responses ranging from \emph{1. mostly true about me} to \emph{4. not true about me}. A higher score was associated with lower feelings about the quality of teamwork for the day. The intraclass correlation for this variable had an absolute value of 0.91. This means that there is a strong correlation between one partners feelings of teamliness and the other partners.

\hypertarget{participants}{%
\subsection{Participants}\label{participants}}

While the original data set included 364 participants of all identities, it predominantly consists of heterosexual individuals (\emph{n} = 353). In order to examine division of labor along gender lines, it was necessary to restrict the data set to male-female couples. Due to the small number of male-female couples that included an individual who was not heterosexual, we decided to restrict our sample further to only include heterosexual couples. Additionally, since all of the couples were living together, we included both couples who were married and those who were in long-term relationships. Regardless of marital status, couples who have lived together for some time need to find ways to divide their housework, and are therefore relevant to our study.

After exclusions and missing data, we were left with 272 individuals (136 dyads). Of these participants, there were 127 married couples, 7 couples that were in committed, long-term relationships, and 2 couples where one individual answered that they were married and the other answered that they were in a long-term relationship. All 136 of the couples answered that they were living together. Notably, 50 participants did not answer any of the demographic information. In these cases, the partner's information was used to calculate these numbers when appropriate. When looking at other household members, 45 of the couples had children under the age of 18 living in their homes. 36 had one child, 31 had two, 13 had three, and 5 had four. Additionally, one couple reported having ten children, and another reported having 26. These two cases may have been participant error when they were inputting the amount of children in their households as text answers. Five couples had one individual report a different number of children than the other. Those who did not answer the question were coded as having no children.

A majority of the participants were White (\emph{n} = 165, 61\%). 24 participants identified as Asian (9\%), 17 identified as Black or African American (6\%), 9 identified as Hispanic or Latinx (3\%), 2 as Middle Eastern (1\%), 3 as White and Hispanic/Latinx (1\%), and 52 either did not answer or chose the ``prefer not to answer'' option (19\%). Nine of the couples were mixed-race couples (6\%). A majority of the individuals identified themselves as Christian (\emph{n} = 150, 55\%). A wide assortment of other religions were also represented, but none had more than 15 people affiliated with them. In terms of political affiliations, there was a pretty equal distribution along the spectrum. This was calculated on a seven point spectrum with 1 equating to far left (liberal), 4 equating to center, and 7 equating to far right (conservative). Most of the individuals were at or around the center of the spectrum, with a slight skew towards conservatism (mean = 4.23, SD = 1.67).

\begin{table}[tbp]

\begin{center}
\begin{threeparttable}

\caption{\label{tab:unnamed-chunk-2}Correlation Matrix Between Sexism Scores}

\begin{tabular}{lllllll}
\toprule
Scales & \multicolumn{1}{c}{Mean} & \multicolumn{1}{c}{SD} & \multicolumn{1}{c}{X1} & \multicolumn{1}{c}{X2} & \multicolumn{1}{c}{X3} & \multicolumn{1}{c}{X4}\\
\midrule
1X Male Hostile Sexism & 3.56 & 0.81 & 1 & .038 & .54 & .3\\
2X Male Benevolent Sexism & 3.95 & 0.70 & .038 & 1 & .3 & .37\\
3X Female Hostile Sexism & 3.23 & 1.01 & .54 & .3 & 1 & .53\\
4X Female Benevolent Sexism & 3.71 & 0.72 & .3 & .37 & .53 & 1\\
\bottomrule
\addlinespace
\end{tabular}

\begin{tablenotes}[para]
\normalsize{\textit{Note.} Possible scores for hostile and benevolent sexism ranged from 1 to 7. All correlations except for male hostile sexism with male benevolent sexism was statistically signficinat with p < .01}
\end{tablenotes}

\end{threeparttable}
\end{center}

\end{table}

\hypertarget{results}{%
\section{Results}\label{results}}

\hypertarget{analysis-strategy}{%
\subsection{Analysis Strategy}\label{analysis-strategy}}

In romantic partnerships, each individual has influence on the other person, and often, mutual influence is also a key part of their relationship. This shared influence violates one of the common assumptions for running statistical models: independence of residuals. In this study, we used the APIM (Kenny et al., 2006) in order to account for the mutual influence that romantic partners may share with each other.

The APIM addresses this interdependence by looking at both actor effects and partner effects. In the model, there are two individuals, each with an independent and a dependent variable. \emph{Actor effects} focus solely on one individual, and describe the effect that the independent variable has on the dependent variable, regardless of the other person. \emph{Partner effects} describe the effect that the other person's independent variable has on this individual's dependent variable. When dyad members have distinguishable characteristics, such as in the case of heterosexual couples, there are two actor effects and two partner effects.

In order to account for the lack of independence in the residuals, the APIM creates two correlations. The first correlation is between the two independent variables, and can be seen as the curved line connecting them on the far left of the figure. The second correlation is the residual non-independence on the dependent variables, which can be seen as the correlation between the errors for both individuals. These correlations and errors allow the APIM to account for the non-independence of the residuals.

In our study, gender was our distinguishing variable, meaning that we focused on the effects of the man and the woman within each dyad. Our explanatory variables were HS and BS, our outcome variable was the feeling of teamliness on any given day, and each individual's percentage of the total daily household chores was a moderating variable. Figure 1 shows a simplified version of our model that doesn't include time and consolidates HS and BS as sexism. Note that in reality, HS and BS would interact with each other both for the actor and the partner, creating a much more complex model.

\hypertarget{results-1}{%
\subsection{Results}\label{results-1}}

Our initial model investigated all of the possible interactions, looking at both the actor and partner effects, and accounting for HS and BS, teamliness, and the percentage of daily household chores (see Table 1). Based on our hypotheses, the main interactions that we wanted to look at were the relationship between the actor and the partner's HS/BS and their feelings about the quality of their relationship and the relationship between the HS/BS of the actor, the percentage of daily chores that they and their partner performed, and their feelings about the quality of their relationship. The significant terms that remained were gender interacted with time, gender interacted with HS of the partner, and gender interacted with the partner's daily percentage of chores performed and the partner's HS. We then began reducing the model by only including the significant terms listed above, as well as the interaction between gender and the partner's daily percentage of chores in order to include the main effects.

This model found that over time, when holding all other variables constant (the man performs no chores), the woman felt less like a team with her partner (p = .0058). It also found the partner effects of HS for the man made his feelings of teamliness higher when his partner was not performing any chores (p = .0265). However, the effect of the interaction between gender, the partner's daily percentage of chores performed, and the partner's HS, which was statistically significant in the previous model, was no longer significant in the reduced model.

Since this interaction was not significant we removed it and ran a new model with only gender, gender interacted with time, gender interacted with the partner's HS, and gender interacted with the partner's daily percentage of chores. This new model confirmed the results from before, but also added new findings: the male partner's effect of daily percent of chores performed decreased the woman's feeling of teamliness (p = .0011).

\begin{tabular}{l|l|r|l|r|r|l}
\hline
\multicolumn{1}{c|}{ } & \multicolumn{3}{c|}{Man} & \multicolumn{3}{c}{Woman} \\
\cline{2-4} \cline{5-7}
Variable & Value\_M & T\_M & P\_M & Value\_W & T\_W & P\_W\\
\hline
Gender & -0.46 & -0.24 & 0.81 & 2.52 & 1.14 & 0.26\\
\hline
Time & >-0.01 & -0.05 & 0.96 & -0.01 & -2.39 & .02*\\
\hline
Partner's Percent of Chores (PC) & 1.43 & 0.76 & 0.45 & -1.01 & -0.45 & 0.65\\
\hline
Actor's PC & 1.86 & 0.97 & 0.33 & -1.51 & -0.69 & 0.49\\
\hline
Actor's Hostile Sexism (HS) & -0.08 & -0.44 & 0.66 & 0.05 & 0.24 & 0.81\\
\hline
Actor's Benevolent Sexism (BS) & 0.49 & 0.98 & 0.33 & 0.38 & 1.14 & 0.26\\
\hline
Partner's HS & 0.44 & 2.32 & .02* & -0.13 & -0.64 & 0.52\\
\hline
Partner's BS & -0.23 & -0.80 & 0.42 & -0.48 & -0.82 & 0.41\\
\hline
Actor's PC x Actor's HS & -0.08 & -0.44 & 0.66 & -0.07 & -0.33 & 0.74\\
\hline
Partner's PC x Actor's HS & 0.11 & 0.65 & 0.51 & 0.24 & 1.05 & 0.39\\
\hline
Actor's PC x Partner's HS & -0.15 & -0.76 & 0.45 & 0.22 & 1.11 & 0.27\\
\hline
Partner's PC x Partner's HS & -0.41 & -2.27 & .02* & -0.02 & -0.10 & 0.92\\
\hline
Actor's PC x Actor's BS & -0.53 & -1.05 & 0.3 & -0.11 & -0.34 & 0.74\\
\hline
Partner's PC x Actor's BS & -0.57 & -1.15 & 0.25 & -0.39 & -1.14 & 0.25\\
\hline
Actor's PC x Partner's BS & 0.18 & 0.63 & 0.55 & 0.36 & 0.63 & 0.53\\
\hline
Partner's PC x Partner's BS & 0.40 & 1.40 & 0.16 & 0.37 & 0.63 & 0.53\\
\hline
\multicolumn{7}{l}{\textsuperscript{*} indicate a p-value lower than our alpha of .05}\\
\end{tabular}

\textbf{Figure 1}

*Representation of our APIM Model

\begin{figure}
\centering
\includegraphics{APIM.png}
\caption{Our APIM Model}
\end{figure}

\emph{Note.}For simplicity, time was not included and HS and BS were simplified as sexism. In reality, HS and BS would interact with each other for both the actor and the partner.

\hypertarget{data-analysis}{%
\subsection{Data analysis}\label{data-analysis}}

We used R {[}Version 4.1.3; R Core Team (2022){]} and the R-packages \emph{dplyr} {[}Version 1.0.8; Wickham, François, Henry, and Müller (2022){]}, \emph{forcats} {[}Version 0.5.1; Wickham (2021){]}, \emph{ggformula} {[}Version 0.10.1; Kaplan and Pruim (2021){]}, \emph{ggplot2} {[}Version 3.3.5; Wickham (2016){]}, \emph{ggridges} {[}Version 0.5.3; Wilke (2021){]}, \emph{ggstance} {[}Version 0.3.5; Henry, Wickham, and Chang (2020){]}, \emph{gtable} {[}Version 0.3.0; Wickham and Pedersen (2019){]}, \emph{kableExtra} {[}Version 1.3.4; Zhu (2021){]}, \emph{lattice} {[}Version 0.20.45; Sarkar (2008){]}, \emph{lubridate} {[}Version 1.8.0; Grolemund and Wickham (2011){]}, \emph{Matrix} {[}Version 1.4.0; Bates and Maechler (2021){]}, \emph{mosaic} {[}Version 1.8.3; Pruim, Kaplan, and Horton (2017); Pruim, Kaplan, and Horton (2021){]}, \emph{mosaicData} {[}Version 0.20.2; Pruim et al. (2021){]}, \emph{nlme} {[}Version 3.1.155; Pinheiro, Bates, DebRoy, Sarkar, and R Core Team (2022){]}, \emph{papaja} {[}Version 0.1.0.9997; Aust and Barth (2020){]}, \emph{psych} {[}Version 2.1.9; Revelle (2021){]}, \emph{purrr} {[}Version 0.3.4; Henry and Wickham (2020){]}, \emph{readr} {[}Version 2.1.2; Wickham, Hester, and Bryan (2022){]}, \emph{stringr} {[}Version 1.4.0; Wickham (2019){]}, \emph{tibble} {[}Version 3.1.6; Müller and Wickham (2021){]}, \emph{tidyr} {[}Version 1.2.0; Wickham and Girlich (2022){]}, and \emph{tidyverse} {[}Version 1.3.1; Wickham et al. (2019){]} for all our analyses.

\newpage

\hypertarget{references}{%
\section{References}\label{references}}

\begingroup
\setlength{\parindent}{-0.5in}
\setlength{\leftskip}{0.5in}

\hypertarget{refs}{}
\begin{CSLReferences}{1}{0}
\leavevmode\vadjust pre{\hypertarget{ref-R-papaja}{}}%
Aust, F., \& Barth, M. (2020). \emph{{papaja}: {Create} {APA} manuscripts with {R Markdown}}. Retrieved from \url{https://github.com/crsh/papaja}

\leavevmode\vadjust pre{\hypertarget{ref-Bareket}{}}%
Bareket, O., Shnabel, N., Kende, A., Knab, N., \& Bar-Anan, Y. (2020). Need some help, honey? Dependency-oriented helping relations between women and men in the domestic sphere. \emph{Journal of Personality and Social Psychology}, \emph{120}. \url{https://doi.org/10.1037/pspi0000292}

\leavevmode\vadjust pre{\hypertarget{ref-Barstad}{}}%
Barstad, A. (2014). Equality is bliss? Relationship quality and the gender division of household labor. \emph{Journal of Family Issues}, \emph{35}(7), 972--992. \url{https://doi.org/10.1177/0192513X14522246}

\leavevmode\vadjust pre{\hypertarget{ref-R-Matrix}{}}%
Bates, D., \& Maechler, M. (2021). \emph{Matrix: Sparse and dense matrix classes and methods}. Retrieved from \url{https://CRAN.R-project.org/package=Matrix}

\leavevmode\vadjust pre{\hypertarget{ref-Bick}{}}%
Bick, B., Alexander, \& Mertens, K. (2020). Work from home after the COVID-19 outbreak, (2017). \url{https://doi.org/10.24149/wp2017r2}

\leavevmode\vadjust pre{\hypertarget{ref-Braun}{}}%
Braun, M., Lewin-Epstein, N., Stier, H., \& Baumgärtner, M. K. (2008). Perceived equity in the gendered division of household labor. \emph{Journal of Marriage and Family}, \emph{70}(5), 1145--1156. Retrieved from \url{http://www.jstor.org/stable/40056333}

\leavevmode\vadjust pre{\hypertarget{ref-Carlson}{}}%
Carlson, M. W., \& Hans, J. D. (2020). Maximizing benefits and minimizing impacts: Dual-earner couples' perceived division of household labor decision-making process. \emph{Journal of Family Studies}, \emph{26}(2), 208--225. \url{https://doi.org/10.1080/13229400.2017.1367712}

\leavevmode\vadjust pre{\hypertarget{ref-Chesters}{}}%
Chesters, J. (2013). Gender convergence in core housework hours: Assessing the relevance of earlier approaches for explaining current trends. \emph{Journal of Sociology}, \emph{49}, 78--96. \url{https://doi.org/10.1177/1440783311427482}

\leavevmode\vadjust pre{\hypertarget{ref-Doan}{}}%
Doan, L., \& Quadlin, N. (2018). Partner characteristics and perceptions of responsibility for housework and child care. \emph{Journal of Marriage and Family}, \emph{81}. \url{https://doi.org/10.1111/jomf.12526}

\leavevmode\vadjust pre{\hypertarget{ref-Erickson}{}}%
Erickson, R. (2005). Why emotion work matters: Sex, gender, and the division of household labor. \emph{Journal of Marriage and Family}, \emph{67}, 337--351. \url{https://doi.org/10.1111/j.0022-2445.2005.00120.x}

\leavevmode\vadjust pre{\hypertarget{ref-Glick1996}{}}%
Glick, P., \& Fiske, S. (1996). The ambivalent sexism inventory: Differentiating hostile and benevolent sexism. \emph{Journal of Personality and Social Psychology}, \emph{70}, 491--512. \url{https://doi.org/10.1037/0022-3514.70.3.491}

\leavevmode\vadjust pre{\hypertarget{ref-Glick2001}{}}%
Glick, P., \& Fiske, S. (2001). An ambivalent alliance: Hostile and benevolent sexism as complementary justifications for gender inequality. \emph{The American Psychologist}, \emph{56}, 109--118. \url{https://doi.org/10.1037/0003-066X.56.2.109}

\leavevmode\vadjust pre{\hypertarget{ref-ASI}{}}%
Glick, P., \& Fiske, S. T. (1997). \emph{Psychology of Women Quarterly}, \emph{21}(1), 119--135. \url{https://doi.org/10.1111/j.1471-6402.1997.tb00104.x}

\leavevmode\vadjust pre{\hypertarget{ref-Greenstein2009}{}}%
Greenstein, T. (2009). National context, family satisfaction, and fairness in the division of household labor. \emph{Journal of Marriage and Family}, \emph{71}, 1039--1051. \url{https://doi.org/10.1111/j.1741-3737.2009.00651.x}

\leavevmode\vadjust pre{\hypertarget{ref-Greenstein1996}{}}%
Greenstein, T. N. (1996). Husbands' participation in domestic labor: Interactive effects of wives' and husbands' gender ideologies. \emph{Journal of Marriage and Family}, \emph{58}(3), 585--595. Retrieved from \url{http://www.jstor.org/stable/353719}

\leavevmode\vadjust pre{\hypertarget{ref-R-lubridate}{}}%
Grolemund, G., \& Wickham, H. (2011). Dates and times made easy with {lubridate}. \emph{Journal of Statistical Software}, \emph{40}(3), 1--25. Retrieved from \url{https://www.jstatsoft.org/v40/i03/}

\leavevmode\vadjust pre{\hypertarget{ref-Hammond}{}}%
Hammond, M. D., \& Overall, N. C. (2013). When relationships do not live up to benevolent ideals: Women's benevolent sexism and sensitivity to relationship problems. \emph{European Journal of Social Psychology}, \emph{43}(3), 212--223. https://doi.org/\url{https://doi.org/10.1002/ejsp.1939}

\leavevmode\vadjust pre{\hypertarget{ref-Helms}{}}%
Helms, H. M., Walls, J. K., Crouter, A. C., \& McHale, S. M. (2010). Provider role attitudes, marital satisfaction, role overload, and housework: A dyadic approach. \emph{Journal of Family Psychology : JFP : Journal of the Division of Family Psychology of the American Psychological Association}, \emph{24 5}, 568--577.

\leavevmode\vadjust pre{\hypertarget{ref-R-purrr}{}}%
Henry, L., \& Wickham, H. (2020). \emph{Purrr: Functional programming tools}. Retrieved from \url{https://CRAN.R-project.org/package=purrr}

\leavevmode\vadjust pre{\hypertarget{ref-R-ggstance}{}}%
Henry, L., Wickham, H., \& Chang, W. (2020). \emph{Ggstance: Horizontal 'ggplot2' components}. Retrieved from \url{https://CRAN.R-project.org/package=ggstance}

\leavevmode\vadjust pre{\hypertarget{ref-Huffman}{}}%
Huffman, A., Matthews, R., \& Irving, L. (2016). Family fairness and cohesion in marital dyads: Mediating processes between work--family conflict and couple psychological distress. \emph{Journal of Occupational and Organizational Psychology}, \emph{90}. \url{https://doi.org/10.1111/joop.12165}

\leavevmode\vadjust pre{\hypertarget{ref-R-ggformula}{}}%
Kaplan, D., \& Pruim, R. (2021). \emph{Ggformula: Formula interface to the grammar of graphics}. Retrieved from \url{https://CRAN.R-project.org/package=ggformula}

\leavevmode\vadjust pre{\hypertarget{ref-Kenny}{}}%
Kenny, D., Kashy, D., Cook, W., \& Simpson, J. (2006). \emph{Dyadic data analysis}. \emph{American Statistician - AMER STATIST} (Vol. 61).

\leavevmode\vadjust pre{\hypertarget{ref-Lam}{}}%
Lam, C. B., McHale, S. M., \& Crouter, A. C. (2012). The division of household labor: Longitudinal changes and within-couple variation. \emph{Journal of Marriage and Family}, \emph{74}(5), 944--952. https://doi.org/\url{https://doi.org/10.1111/j.1741-3737.2012.01007.x}

\leavevmode\vadjust pre{\hypertarget{ref-Lee}{}}%
Lee, T., Fiske, S., Glick, P., \& Chen, Z. (2010). Ambivalent sexism in close relationships: (Hostile) power and (benevolent) romance shape relationship ideals. \emph{Sex Roles}, \emph{62}, 583--601. \url{https://doi.org/10.1007/s11199-010-9770-x}

\leavevmode\vadjust pre{\hypertarget{ref-Minnotte}{}}%
Minnotte, K., Minnotte, M., \& Pedersen, D. (2013). Marital satisfaction among dual‐earner couples: Gender ideologies and family‐to‐work conflict. \emph{Family Relations}, \emph{62}. \url{https://doi.org/10.1111/fare.12021}

\leavevmode\vadjust pre{\hypertarget{ref-R-tibble}{}}%
Müller, K., \& Wickham, H. (2021). \emph{Tibble: Simple data frames}. Retrieved from \url{https://CRAN.R-project.org/package=tibble}

\leavevmode\vadjust pre{\hypertarget{ref-Nordenmark}{}}%
Nordenmark, M., \& Nyman, C. (2003). Fair or unfair? Perceived fairness of household division of labour and gender equality among women and men the swedish case. \emph{European Journal of Women's Studies}, \emph{10}, 181--209. \url{https://doi.org/10.1177/1350506803010002004}

\leavevmode\vadjust pre{\hypertarget{ref-R-nlme}{}}%
Pinheiro, J., Bates, D., DebRoy, S., Sarkar, D., \& R Core Team. (2022). \emph{{nlme}: Linear and nonlinear mixed effects models}. Retrieved from \url{https://CRAN.R-project.org/package=nlme}

\leavevmode\vadjust pre{\hypertarget{ref-Poortman}{}}%
Poortman, A.-R., \& Lippe, T. V. D. (2009). Attitudes toward housework and child care and the gendered division of labor. \emph{Journal of Marriage and Family}, \emph{71}(3), 526--541. Retrieved from \url{http://www.jstor.org/stable/40262900}

\leavevmode\vadjust pre{\hypertarget{ref-R-mosaic}{}}%
Pruim, R., Kaplan, D. T., \& Horton, N. J. (2017). The mosaic package: Helping students to 'think with data' using r. \emph{The R Journal}, \emph{9}(1), 77--102. Retrieved from \url{https://journal.r-project.org/archive/2017/RJ-2017-024/index.html}

\leavevmode\vadjust pre{\hypertarget{ref-R-mosaicData}{}}%
Pruim, R., Kaplan, D., \& Horton, N. (2021). \emph{mosaicData: Project MOSAIC data sets}. Retrieved from \url{https://CRAN.R-project.org/package=mosaicData}

\leavevmode\vadjust pre{\hypertarget{ref-R-base}{}}%
R Core Team. (2022). \emph{R: A language and environment for statistical computing}. Vienna, Austria: R Foundation for Statistical Computing. Retrieved from \url{https://www.R-project.org/}

\leavevmode\vadjust pre{\hypertarget{ref-R-psych}{}}%
Revelle, W. (2021). \emph{Psych: Procedures for psychological, psychometric, and personality research}. Evanston, Illinois: Northwestern University. Retrieved from \url{https://CRAN.R-project.org/package=psych}

\leavevmode\vadjust pre{\hypertarget{ref-R-lattice}{}}%
Sarkar, D. (2008). \emph{Lattice: Multivariate data visualization with r}. New York: Springer. Retrieved from \url{http://lmdvr.r-forge.r-project.org}

\leavevmode\vadjust pre{\hypertarget{ref-Thomae}{}}%
Thomae, M., \& Houston, D. (2016). The impact of gender ideologies on men's and women's desire for a traditional or non-traditional partner. \emph{Personality and Individual Differences}, \emph{95}, 152--158. \url{https://doi.org/10.1016/j.paid.2016.02.026}

\leavevmode\vadjust pre{\hypertarget{ref-R-ggplot2}{}}%
Wickham, H. (2016). \emph{ggplot2: Elegant graphics for data analysis}. Springer-Verlag New York. Retrieved from \url{https://ggplot2.tidyverse.org}

\leavevmode\vadjust pre{\hypertarget{ref-R-stringr}{}}%
Wickham, H. (2019). \emph{Stringr: Simple, consistent wrappers for common string operations}. Retrieved from \url{https://CRAN.R-project.org/package=stringr}

\leavevmode\vadjust pre{\hypertarget{ref-R-forcats}{}}%
Wickham, H. (2021). \emph{Forcats: Tools for working with categorical variables (factors)}. Retrieved from \url{https://CRAN.R-project.org/package=forcats}

\leavevmode\vadjust pre{\hypertarget{ref-R-tidyverse}{}}%
Wickham, H., Averick, M., Bryan, J., Chang, W., McGowan, L. D., François, R., \ldots{} Yutani, H. (2019). Welcome to the {tidyverse}. \emph{Journal of Open Source Software}, \emph{4}(43), 1686. \url{https://doi.org/10.21105/joss.01686}

\leavevmode\vadjust pre{\hypertarget{ref-R-dplyr}{}}%
Wickham, H., François, R., Henry, L., \& Müller, K. (2022). \emph{Dplyr: A grammar of data manipulation}. Retrieved from \url{https://CRAN.R-project.org/package=dplyr}

\leavevmode\vadjust pre{\hypertarget{ref-R-tidyr}{}}%
Wickham, H., \& Girlich, M. (2022). \emph{Tidyr: Tidy messy data}. Retrieved from \url{https://CRAN.R-project.org/package=tidyr}

\leavevmode\vadjust pre{\hypertarget{ref-R-readr}{}}%
Wickham, H., Hester, J., \& Bryan, J. (2022). \emph{Readr: Read rectangular text data}. Retrieved from \url{https://CRAN.R-project.org/package=readr}

\leavevmode\vadjust pre{\hypertarget{ref-R-gtable}{}}%
Wickham, H., \& Pedersen, T. L. (2019). \emph{Gtable: Arrange 'grobs' in tables}. Retrieved from \url{https://CRAN.R-project.org/package=gtable}

\leavevmode\vadjust pre{\hypertarget{ref-R-ggridges}{}}%
Wilke, C. O. (2021). \emph{Ggridges: Ridgeline plots in 'ggplot2'}. Retrieved from \url{https://CRAN.R-project.org/package=ggridges}

\leavevmode\vadjust pre{\hypertarget{ref-R-kableExtra}{}}%
Zhu, H. (2021). \emph{kableExtra: Construct complex table with 'kable' and pipe syntax}. Retrieved from \url{https://CRAN.R-project.org/package=kableExtra}

\end{CSLReferences}

\endgroup


\end{document}
